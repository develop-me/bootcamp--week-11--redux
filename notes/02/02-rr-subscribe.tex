First things first, let's break up our iCounter app into a few separate components. This is not strictly necessary, but it will make the code examples more like an actual app made up of lots of components.
\\

We're going to put all our JSX files into a directory called \texttt{components}, this will keep the \texttt{src} directory clean. First move \texttt{App} into \texttt{src/components} and make sure you update the \texttt{import} in \texttt{index}.
\\

We'll create a \texttt{<Value>} component in \texttt{src/components/Value/index.js}:

\js{}{02/figures/02/01-Value}

And a \texttt{<Buttons>} component in \texttt{src/components/Buttons/index.js}:

\js{}{02/figures/02/02-Buttons}


\begin{infobox}{Importing \texttt{index}}
    If we have a file called \texttt{index.js} in a directory then we can import it using \texttt{import} just by using the directory name, just as you can with an \texttt{index.html} file.
\end{infobox}

We can then update the \texttt{<App>} component to use it:

\js{}{02/figures/02/03-App}



\section{Connect}

Now we can use React Redux's \texttt{connect()} function to join up our components with the store. This allows us to get the current state and pass it into our component. It will also \texttt{subscribe} to the store for us to re-render the component whenever the state changes.
\\

We'll need to create a \textbf{container component} which uses the \texttt{connect} function.
\\

First, rename your \texttt{<Value>} component from \texttt{src/components/Value/index.js} to \texttt{src/components/Value/Value.js} (you'll see why in a second). This will break your app temporarily, but we'll fix it shortly.
\\

Now create a new file \texttt{src/components/Values/index}. This is going to hold the container component.
\\

The idea of a container component is to wrap around a regular React component and pass in some props. This is useful because the container component can use the \texttt{connect} function to get values from the store. It returns a regular React component that we can use elsewhere.

\js{}{02/figures/02/04-Value-index}

First we import the \texttt{connect} function. This talks to the \texttt{<Provider>} wrapper that we put around the entire app and which has access to the Redux store. We also import in the component that we're interested in giving access to the store, in this case the \texttt{<Value>} component.
\\

Then we create a function called \texttt{mapStateToProps}.\footnote{You can call it whatever you like, it's just a variable that we pass to \texttt{connect}, but it's pretty standard to stick to this name.} We're going to pass this function to \texttt{connect}, which will then call it for us whenever the Redux store updates the state (using \texttt{store.subscribe()}). When this happens it will pass in the latest version of the state and merge whatever is returned into the props for the wrapped component.
\\


\img{12cm}{02/img/02/mapStateToProps-before}{2em}{Before: no way to access the Redux store}

\par\bigskip

\img{14cm}{02/img/02/mapStateToProps-after}{2em}{After: wrap with a component that has access to the Redux store}

\pagebreak

\begin{infobox}{Working with Objects}
    We could actually have written \texttt{mapStateToProps} like this:

    \begin{minted}{jsx}
        const mapStateToProps = ({ count }) => ({
            count,
        });
    \end{minted}

    This is just using \texttt{=>} to return a single value. However, we need to use the brackets around the object literal, otherwise the curly braces are interpreted as opening a function block.
    \\

    We've also used destructuring to get the \texttt{count} property from \texttt{state} and taken advantage of the fact that if an object key and variable name match, we can just pass the variable and JS will add the matching key for us (known as \href{https://ariya.io/2013/02/es6-and-object-literal-property-value-shorthand}{shorthand properties}).
\end{infobox}

Now we can update \texttt{index.js} to not pass the \texttt{count} prop in:

\js{}{02/figures/02/05-index}

And we can also update \texttt{<App>} to not deal with the \texttt{count} prop (as it doesn't get passed in now anyway):

\js{}{02/figures/02/06-App}


\section{Own Props}

Sometimes it can still be useful to pass props into wrapped component. We can do this just as before:

\begin{minted}{jsx}
    <Value offset={ 5 } />
\end{minted}

Any props passed to a wrapped component get passed through to the underlying component:

\begin{minted}{jsx}
    const Value = ({
      count,
      offset,
    }) => (
      <p className="card card-body">{ count + offset }</p>
    );
\end{minted}

You can also access props inside the container component, as they are passed in as the second argument to \texttt{mapStateToProps}:

\begin{minted}{jsx}
    const mapStateToProps = (state, { offset }) => {
      return {
        count: state.count + offset,
      };
    };
\end{minted}

This can save passing values all the way into the wrapped component.


\section{Additional Resources}

\begin{itemize}[leftmargin=*]
    \item \href{https://react-redux.js.org/using-react-redux/connect-mapstate}{React Redux: \texttt{mapStateToProps}}
    \item \href{https://medium.com/mofed/reduxs-mysterious-connect-function-526efe1122e4}{Redux’s Mysterious Connect Function}
    \item \href{https://blog.bitsrc.io/6-tricks-with-resting-and-spreading-javascript-objects-68d585bdc83}{7 Tricks with Resting and Spreading JavaScript Objects}
    \item \href{https://www.robinwieruch.de/react-folder-structure}{React Folder Structure}
\end{itemize}
